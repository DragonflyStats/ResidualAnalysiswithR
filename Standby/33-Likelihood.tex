\documentclass[residuals.tex]{subfiles}
\begin{document}
	
\newpage
\subsection{Likelihood}	
	
Likelihood is the probability of seeing the data you collected given your model. 
The logic of the likelihood ratio test is to compare the likelihood of two models 
with each other. 

%============================================================================= %

First, the model without the factor that you’re interested in (the 
null model), then the model with the factor that you’re interested in. Maybe an 
analogy helps you to wrap your head around this: Say, you’re a hiker, and you 
carry a bunch of different things with you (e.g., a gallon of water, a flashlight). 

%============================================================================= %

To know whether each item affects your hiking speed, you need to get rid of it. So, 
you get rid of the flashlight and run without it. Your hiking speed is not affected 
much. Then, you get rid of the gallon of water, and you realize that your hiking 
speed is affected a lot. 

You would conclude that carrying a gallon of water with 
you significantly affects your hiking speed whereas carrying a flashlight does not.
Expressed in formula, you would want to compare the following two “models” 
(think “hikes”) with each other:

%============================================================================= %

\begin{verbatim}
mdl1 = hiking speed ~ gallon of water + flashlight
mdl2 = hiking speed ~ flashlight
\end{verbatim}
If there is a significant difference between “mdl2” and “mdl1”, then you know 
that the gallon of water matters. To assess the effect of the flashlight, you would 
have to do a similar comparison:
\begin{verbatim}
mdl1 = hiking speed ~ gallon of water + flashlight
mdl2 = hiking speed ~ gallon of water
\end{verbatim}
In both cases, we compared a full model (with the fixed effects in question) 
against a reduced model without the effects in question. In each case, we conclude 
that a fixed effect is significant if the difference between the likelihood of these 
two models is significant. 

%=============================================================================== %
\end{document}