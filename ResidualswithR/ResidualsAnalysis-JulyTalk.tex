% !TEX TS-program = pdflatex
% !TEX encoding = UTF-8 Unicode

% This is a simple template for a LaTeX document using the "article" class.
% See "book", "report", "letter" for other types of document.

\documentclass[11pt]{article} % use larger type; default would be 10pt

\usepackage[utf8]{inputenc} 
\usepackage{geometry} % to change the page dimensions
\geometry{a4paper} 
\usepackage{graphicx} 
\usepackage{amsmath}
\usepackage{framed}
\usepackage{amssymb}
\usepackage{booktabs} % for much better looking tables
\usepackage{array} % for better arrays (eg matrices) in maths
\usepackage{paralist} % very flexible & customisable lists (eg. enumerate/itemize, etc.)
\usepackage{verbatim}
\usepackage{subfig} 
\usepackage{fancyhdr} % This should be set AFTER setting up the page geometry
\pagestyle{fancy} % options: empty , plain , fancy
\renewcommand{\headrulewidth}{0pt} % customise the layout...
\lhead{Dublin \texttt{R}}\chead{Cluster Analysis with \texttt{R}}\rhead{June 2014}
\lfoot{}\cfoot{\thepage}\rfoot{}

%%% SECTION TITLE APPEARANCE
\usepackage{sectsty}
\allsectionsfont{\sffamily\mdseries\upshape} % (See the fntguide.pdf for font help)
% (This matches ConTeXt defaults)

%%% ToC (table of contents) APPEARANCE
\usepackage[nottoc,notlof,notlot]{tocbibind} % Put the bibliography in the ToC
\usepackage[titles,subfigure]{tocloft} % Alter the style of the Table of Contents
\renewcommand{\cftsecfont}{\rmfamily\mdseries\upshape}
\renewcommand{\cftsecpagefont}{\rmfamily\mdseries\upshape} % No bold!

%%% END Article customizations

%%% The "real" document content comes below...

\title{Residual Analysis with \texttt{R}}
\author{The Author}
%\date{} % Activate to display a given date or no date (if empty),
         % otherwise the current date is printed 

\begin{document}
\tableofcontents

\newpage
\section{Residual Analysis for Linear Models}

\subsection{Key Concepts}

\begin{description}
\item[Residual]

\item[Influence]

\item[Outlier]

\item[Leverage]

\end{description}

\subsection{Studentized and Standardized Residuals}
%---------------------------------------------------------------------------%
\newpage


\subsection{Cook's Distance}

\subsubsection{\texttt{infleunce}}
This function provides the basic quantities which are used in forming a wide variety of diagnostics for checking the quality of regression fits.



\subsubsection{Interpreting Cook's Distance}


%---------------------------------------------------------------------------%
\subsubsection{Using the \texttt{plot} command}

\begin{itemize}
\item Homogenety of Variance of Residuals
\item 
\item
\end{itemize}


\subsubsection{Coding the four plots together}

\begin{framed}
\begin{verbatim}
par(mfrow=c(2,2))

plot(FittedModel)

par(opar)
\end{verbatim}
\end{framed}

%---------------------------------------------------------------------------%
\newpage
\subsection{Diagnostics of LME Models}

% Schabenberger
% Zewotir
% Haslett Hayes





%---------------------------------------------------------------------------%

\end{document}