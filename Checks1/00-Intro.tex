\documentclass[main.tex]{subfiles}

% - http://strata.uga.edu/6370/rtips/regressionPlots.html

\begin{document}
	\subsection{Residual Plots}
	A residual plot is a graph that shows the residuals on the vertical axis and the independent variable on the horizontal axis. If the points in a residual plot are randomly dispersed around the horizontal axis, a linear regression model is appropriate for the data; otherwise, a non-linear model is more appropriate.
	
	Below the table on the left shows inputs and outputs from a simple linear regression analysis, and the chart on the right displays the residual (e) and independent variable (X) as a residual plot.
	
	\begin{center}
		\begin{tabular}{|c|c|c|c|c|c|}
			x &	60	& 70	& 80	& 85 &	95 \\ \hline
			y &	70	& 65	& 70	& 95 &	85 \\ \hline
			y.hat	& 65.411 &	71.849 &	78.288 &	81.507	& 87.945 \\ \hline
			e	& 4.589	& -6.849 &	-8.288 &	13.493 &	-2.945 \\ \hline
		\end{tabular}
	\end{center}
	
	The residual plot shows a fairly random pattern - the first residual is positive, the next two are negative, the fourth is positive, and the last residual is negative. This random pattern indicates that a linear model provides a decent fit to the data.
	
	Below, the residual plots show three typical patterns. The first plot shows a random pattern, indicating a good fit for a linear model. The other plot patterns are non-random (U-shaped and inverted U), suggesting a better fit for a non-linear model.
	\newpage
	\Large
	%--------------------------------------------------------------------------------------%
	\section{Model Validation}
	%http://www.itl.nist.gov/div898/handbook/pmd/section4/pmd44.htm
	\begin{itemize}
		\item Model validation is possibly the most important step in the model building sequence. It is also one of the most overlooked. Often the validation of a model seems to consist of nothing more than quoting the $R^2$ statistic from the fit (which measures the fraction of the total variability in the response that is accounted for by the model). 
		
		\item Unfortunately, a high $R^2$ value does not guarantee that the model fits the data well. Use of a model that does not fit the data well cannot provide good answers to the underlying engineering or scientific questions under investigation.
		
		
		
		\item Model diagnostic techniques determine whether or not the distributional assumptions are satisfied, and to assess the influence of unusual observations.
	\end{itemize}
	\bigskip
	
	\subsection{Why Use Residuals?}
	
	If the model fit to the data were correct, the residuals would approximate the random errors that make the relationship between the explanatory variables and the response variable a statistical relationship. 
	
	
	Therefore, if the residuals appear to behave randomly, it suggests that the model fits the data well. On the other hand, if non-random structure is evident in the residuals, it is a clear sign that the model fits the data poorly. 
	
	The subsections listed below detail the types of plots to use to test different aspects of a model and give guidance on the correct interpretations of different results that could be observed for each type of plot.
	%%------------------------------------------------------------------------------------------------------------------------ %
	%\section{Introduction to Residuals}
	%
	%The difference between the observed value of the dependent variable (y) and the predicted value ($\hat{y}$) is called the \textbf{residual} (e). Each data point has one residual.
	%
	%\[\mbox{Residual} = \mbox{Observed value} - \mbox{Predicted value}\] 
	%\[e = y - \hat{y}\]
	%
	%Both the sum and the mean of the residuals are equal to zero. 
	%%That is, Σ e = 0 and e = 0.
	
	%\subsection{Residual Plots}
	%A residual plot is a graph that shows the residuals on the vertical axis and the independent variable on the horizontal axis. If the points in a residual plot are randomly dispersed around the horizontal axis, a linear regression model is appropriate for the data; otherwise, a non-linear model is more appropriate.
\end{document}
